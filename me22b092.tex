\documentclass{article}
\usepackage[utf8]{inputenc}
\title{me22b092.tex}
\author{Abhinav Kumar}
\date{4th February 2023}

\begin{document}

\maketitle

\section{Introduction of Couloumb's Law ME22B092}
Coulomb's inverse-square law, or simply Coulomb's law, is an experimental law of physics that quantifies the amount of force between two stationary, electrically charged particles. The electric force between charged bodies at rest is conventionally called electrostatic force or Coulomb force.Although the law was known earlier, it was first published in 1785 by French physicist Charles-Augustin de Coulomb, hence the name. Coulomb's law was essential to the development of the theory of electromagnetism, maybe even its starting point, as it made it possible to discuss the quantity of electric charge in a meaningful way.

The law states that the magnitude of the electrostatic force of attraction or repulsion between two point charges is directly proportional product of the magnitudes of charges and inversely proportional to the square of the distance between them. Coulomb studied the repulsive force between bodies having electrical charges of the same sign, and the attractive force between charges having different sign.

\section{Couloumb Law Equation }
\begin{equation}
  {\bf \vec F}\footnote{This equation I referred from wikipedia link: \url{https://en.wikipedia.org/wiki/Coulomb\%27s\_law}}={\bf {q_1.q_2 \over 4\pi\epsilon r^3} \hat{r} }
\end{equation}
where,

     $q_1=magnitude\;of\;first\;charge$
     
     $q_2=magnitude\;of\;second\;charge$
    
     $\epsilon=permitivity\;of\;the\;medium\;in\;which\;charges\;are\;placed$
    
     $r=distance\;between\;the\;charges$ \\

\emph{Git-id:Abhinav-Kumar-ME22B092}
\end{document}
